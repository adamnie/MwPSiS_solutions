\section{Wstęp}

Sieci sterowane programowo (SDN, z ang. Software Defined Networks) to koncept który coraz bardziej zyskuje na wadze w dzisiejszym świecie architektury systemów i sieci. Koncepcja oddzielenia warstwy kontrolującej sieć od mechanizmów związanych z transmisją danych rozwiązuje wiele problemów przed którymi stają codziennie architekci sieci. Jednym z wyzwań przed którym stają osoby chcące zaimplementować w swojej sieci mechanizm SDN jest kontrola parametrów wśród . Jest to krytycznie ważna kwestia ponieważ w rzeczywistych zastosowaniach, każdy z klientów (dzierżawców) sieci ma swoje wymagania odnośnie świadczonych przez operatora usług, których nie spełnienie może wiązać się z poważnymi konsekwencjami finansowymi.
W pracy \cite{lin16} autorzy proponują kompleksowe rozwiązanie tego problemu.
