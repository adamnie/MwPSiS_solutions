\section{Wstęp}

Sieci sterowane programowo (SDN, z ang. Software Defined Networks) to koncept, który coraz bardziej zyskuje na wadze w dzisiejszym świecie architektury systemów i sieci. Koncepcja oddzielenia warstwy kontrolującej sieć od mechanizmów związanych z transmisją danych rozwiązuje wiele problemów, przed którymi stają codziennie architekci sieci. Jednym z wyzwań, przed którym stają operatorzy chcący zaimplementować w swojej sieci mechanizm SDN jest sprawiedliwy podział zasobów pomiędzy swoich klientów (dzierżawców) oraz zapewnienie takiej alokacji przepływów, która zapewni płynne działanie sieci. Jest to krytycznie ważna kwestia, ponieważ w rzeczywistych zastosowaniach, każdy z klientów (dzierżawców) sieci ma swoje wymagania odnośnie świadczonych przez operatora usług, których niespełnienie może wiązać się z poważnymi konsekwencjami finansowymi. Do tego typu parametrów zaliczamy jitter, opóźnienie przesyłu czy utratę pakietów. \newline

\noindent W pracy \cite{lin16} autorzy proponują kompleksowe rozwiązanie powyższego problemu, obejmujące dwustopniowy model optymalizacyjny (wirtualizacja zasobów poprzez podział sieci oraz maksymalizacja minimalnego przepływu).  \newline

\noindent W poniższej pracy autorzy skupili się na optymalizacji sposobu alokacji przepływów względem innych parametrów niż te zaproponowane w \cite{lin16}, co może bardziej realistycznie odpowiadać zapotrzebowaniom klientów. W tym celu przeanalizowano algorytm zaproponowany w \cite{lin16}, dokonano jego implementacji w języku ILOG, a także rozszerzono go, proponując dwa alternatywne modele ostatniej fazy działania algorytmu.
