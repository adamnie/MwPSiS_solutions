\section{Rozszerzenie}

Po zaimplementowaniu podstawowego modelu postanowiliśmy podejść do tematu
rozszerzenia problemu na dwa sposoby: dodaniu nowych ograniczeń
oraz zastosowaniu go w innej formie. Zaproponowanym przez nas ograniczeniem jest
zapewnianie minimaalnej wartości przepływu, a nowym zastosowaniem jest wprowadzenie
kosztu jako głównej metryki optymalizacji sieci.

\subsection{Zapewnienie minimalnych wartości przepływu}

Istotnym aspektem nie poruszanym przez autorów pracy jest zapewnienie
minimalnej wartości przepływu. Autorzy odnieśli się do problemu zapewnienia wszystkim
przepływom dodatniej wartości (w praktyce wszystie przepływy będą miały równą wartość).
Pozwala to na sprawiedliwy podział zasobów, natomiast w przypadku jeżeli różnym
przepływom chcemy zapewnić różne minimalne wartości potrzebne jest rozszerzenie problemu
o nowe dane (minimalną liczbę danych dla każdego przepływu) oraz dodatkowe ograniczenie:

\begin{equation}
  \forall_{t \in T} \forall_{f \in F_t} \quad \lambda_{tf} \ge f.minimal
\end{equation}

\subsection{Koszt przesyłu danych}

W przypadku nowego zastosowania postanowiliśmy postawić na praktyczne podejście.
Rozszerzylismy model o koszt przesyłanych danych i zmieniliśmy funkcję celu tak,
by zrównoważyć koszt przepływów:
\begin{equation}
  \sum_{t \in T} \max_{f \in F_t} \sum_{a \in A_f} a.cost
\end{equation}

\subsection{Wyniki roszerzonych modeli}
Poniżej przedstawiamy porównanie wyników modeli rozszerzonych i podstawowego wyliczonych
w środowisku CPLEX. Tabela zawiera wyniki dla trzech zestawów danych: sieci małej (4 węzły),
średniej (6 węzłów) i dużej (17 węzłów).

\begin{table}[!h]
\centering
\caption{Mała topologia}
\label{my-label}
\begin{tabular}{|l|c|c|c|}
\hline
Mała topologia      & \multicolumn{1}{l|}{model podstawowy} & \multicolumn{1}{l|}{model z minimalnym przepływem} & \multicolumn{1}{l|}{model z minimalnym kosztem} \\ \hline
Jitter              & 0                                     & 0                                                  & 0                                               \\ \hline
Delay               & 2,9                                   & 2,9                                                & 4,83                                            \\ \hline
Sumaryczny przepływ & 10                                    & 10                                                 & 5                                               \\ \hline
czas wykonania      & 31                                    & 37                                                 & 22                                              \\ \hline
\end{tabular}
\end{table}

\begin{table}[!h]
\centering
\caption{Średnia topologia}
\label{my-label}
\begin{tabular}{|l|c|c|c|}
\hline
Średnia topologia   & \multicolumn{1}{l|}{model podstawowy} & \multicolumn{1}{l|}{model z minimalnym przepływem} & \multicolumn{1}{l|}{model z minimalnym kosztem} \\ \hline
Jitter              & 0,0023                                & 0,0004                                             & 0                                               \\ \hline
Delay               & 1,92921099                            & 2,45                                               & 3,3                                             \\ \hline
Sumaryczny przepływ & 50                                    & 50                                                 & 2                                               \\ \hline
czas wykonania      & 71                                    & 88                                                 & 89                                              \\ \hline
\end{tabular}
\end{table}

\begin{table}[!h]
\centering
\caption{Duża topologia}
\label{my-label}
\begin{tabular}{|l|c|c|c|}
\hline
Duża topologia      & \multicolumn{1}{l|}{model podstawowy} & \multicolumn{1}{l|}{model z minimalnym przepływem} & \multicolumn{1}{l|}{model z minimalnym kosztem} \\ \hline
Jitter              & 6,51E-05                              & 6,51E-05                                           & 4,93E-32                                        \\ \hline
Delay               & 3,57498538                            & 3,82498538                                         & 4,83                                            \\ \hline
Sumaryczny przepływ & 80                                    & 80                                                 & 5                                               \\ \hline
czas wykonania      & 917096                                & 908099                                             & 908099                                          \\ \hline
\end{tabular}
\end{table}

\subsection{Interpretacja wyników}

Jak wynika z danych zamieszczonych w powyższych tabelach model podstawowy i model z zapewnioną
wartością minimalnego przepływu dają bardzo podobne wyniki wszystkich parametrów. Są one bardzo
satysfakcjonujące ponieważ nasz model ma taką samą efektywność jak model oryginalny, daje natomiast
dodatkową elastyczność poprzez zróżnicowanie wymagań dla poszczególnych przepływów. Pewne różnice można
zaobserwować pomiędzy wartościami parametrów dla modelu zorientowanego na koszt a tymi, które maksymalizują
wartość przepływu. Jest to oczywiście rezultat oczekiwany, gdyż koszt i przeływność w takim ujęciu
są proporcjonalne, wobec tego gdy minimalizując koszt minimalizujemy również wartość przepływu. Dane potwierdzają więc,
że model minimalizujący koszty przepływów działa tak jak zakładano.
