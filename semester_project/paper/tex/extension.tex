\section{Rozszerzenie}

Po zaimplementowaniu podstawowego modelu postanowiliśmy podejść do tematu
rozszerzenia podstawowego problemu na dwa sposoby dodaniu nowych ograniczeń
oraz zastosowaniu go w innej formie. Zaproponowanym przez nas ograniczeniem jest
zapewnianie minimlanej wartości przepływu a nowym zastosowaniem jest wprowadzenie
kosztu jako głównej metryki decydowania o wykonalności problemu.

\subsection{Zapewnienie minimalnych wartości przepływu}

Istotnym aspektem nie poruszanym przez autorów pracy jest zapewnienie
minimalnej wartości przepływu. Choć autorzy odnieśli się do problemu "zagłodzenia"
ruchu poprzez maksymallizację minimalnego przepływu, czyli w praktyce zrównoważeniu
podziału zasobów pomiędzy przepływy. Natomiast w przypadku jeżeli różnym przepływom
chcemy zapewnić różne minimalne wartości potrzebne jest rozszerzenie problemu o
nowe dane (minimalną liczbę danych per przepływ) oraz dodatkowe ograniczenie:

\begin{equation}
  \forall_{t \in T} \forall_{f \in F_t} \quad \lambda_{tf} \ge f.minimal
\end{equation}

\subsection{Koszt przesyłu danych}

W przypadku nowego zastosowania postanowiliśmy postawić na praktyczne podejście;
rozszerzylismy model o koszt przesyłanych danych i zmieniliśmy funkcję celu tak,
by zrównoważyć koszt przepływów:
\begin{equation}
  \sum_{t \in T} \max_{f \in F_t} \sum_{a \in A_f} a.cost
\end{equation}

\subsection{Wyniki roszerzonych modeli}
Poniżej przedstawiamy porównanie wyników modeli rozszerzonych i podstawowego wyliczonych
w środowisku CPLEX. Tabela zawiera wyniki dla trzech zestawów danych: sieci małej (4 hosty),
średniej (6 hostów) i dużej (17 hostów).
