\section{Model alokacji przepływu}

Naszym kolejnym krokiem było zaimplementowanie modelu alokacji przepływów, w
którym funkcją celu jest maksymalizacja minimalnego przepływu, czyli zrównoważenie
ruchu generowanego przez przepływy dzierżawców na wszystkie dostępne łącza:

\begin{equation}
  \sum_{n \in N} \max_{\{X_{f,i,j}^n;\forall{(i,j)} \in E_n^f,f \in F^n\}} \min_{f \in F^n}\lambda_f^n
\end{equation}

W modelu zostało zaimplementowanych pięć ograniczeń. Pierwsze z nich zapewnia,
że suma wartości przepływów na danym łączu nie przekroczy jego pojemności:

\begin{equation}
  \sum_{n \in N} \sum_{f \in F^n} X_{f,i,j}^n \le c_{ij}
\end{equation}

Kolene ograniczenie nazywane konserwacją przepływów zapewnia, że dla danego węzła
nie należącego do węzła docelowego dla dowolnego przepływu suma wartości wychodzących
przepływów jest równa sumie wartości przychodzących przepływów plus wartość danego
przepływu jeśli jest to węzęł źródłowy dla tego przepływu. W innym wypadku suma ta
równa się zero:

\begin{equation}
\begin{align*}
  \sum_{j;(i,j) \in E_n^f} X_{f,i,j}^n &- \sum_{j;(j,i) \in E_n^f} X_{f,i,j}^n = \lambda_f^n \parallel_{\{i=s_f^n\}} \\
  & \forall{i} \ne d_f^n, f \in F^n, n \in N
\end{align*}
\end{equation}

Następne ograniczenie zapewnia, że przepływ nie może zostać rozdzielony na
dwa lub więcej łączy:

\begin{equation}
  X_{f,i,j}^n = 0 \quad \forall(i,j) \notin E_n^f, f \in F^n, n \in N
\end{equation}

Czwarte ograniczenie związane jest z zapewnieniem jakości obsługi. Zapewnia ono, że
iloczyn straty pakietów na łączach przypisanych do danego przepływu nie przekroczy
ustalonej maksymalnej wartości:

\begin{equation}
  \prod_{(i,j) \in E_n^f} p_{f,i,j}^n < p_{n,f}^{max}
\end{equation}

Ostatnie ograniczenie nie jest opisane przez autorów, zostało dodane przez nas
na potrzeby naszej struktury danych wejściowych a dokładnie podzielonego grafu pomiędzy
dzierżawców. Ograniczenie to zapewnia, że dla danego dzierżawcy, każdy jego przepływ
nie będzie wykorzystywał łącz, które nie są jego:

\begin{equation}
  \forall_{f \in F^n} X_{f,i,j} = 0 \quad \text{jeśli}\quad x_{i,j}^f = 0
\end{equation}
