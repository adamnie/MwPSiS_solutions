\documentclass{article}
\usepackage[utf8]{inputenc}

\title{Homework 5}
\author{Adam Niedziałkowski}
\date{26 December 2016}

\begin{document}

\maketitle

\section{Problem}
Projektowanie szerokopasmowej sieci dostępowej można przedstawić następująco
(zapis jest celowo nadmiarowy): w pewnej lokalizacji między cen-
tralą a grupą klientów instaluje się węzeł pośredniczący, do którego od cen-
trali doprowadza się kabel światłowodowy, a potem od niego rozprowadza
sygnał za pomocą kabli miedzianych do klientów (np. z użyciem techniki
xDSL).  Węzeł  pośredniczący  dokonuje  konwersji  optyczno-elektrycznej  i
pracuje  jako  koncentrator.  Użycie  jak  najkrótszego  segmentu  złożonego
z  kabli  miedzianych  byłoby  korzystne  dla  klienta,  ponieważ  im  krótszy
taki segment, tym większa przepływność, ale z punktu widzenia operatora
sensowne jest użycie jak nadłuższych odcinków już dawno położonej in-
frastruktury  miedzianej  (w  związku  z  użyciem  istniejącej  infrastruktury
pomijamy  tutaj  koszty  położenia  kabli).  Przy  założonej  przepływności,
którą ma uzyskać każdy klient, długość okablowania miedzianego łączą-
cego  węzeł  pośredniczący  z  klientem  nie  może  być  dłuższa  niż R km.
Z punktu widzenia topologii fizycznej sieć złożona z wierzchołków reprezen-
tujących centralę, węzeł pośredniczący (węzły pośredniczące) oraz klien-
tów jest drzewem. Problem polega na znalezieniu takiego umiejscowienia
węzłów pośredniczących obsługujących wszystkich klientów, że pojedyn-
czy  węzeł  pośredniczący  może  obsłużyć  wszystkich  przyłączonych  klientów.

\subsection{Oznaczenia}
Oznaczenia:

$S - $ zbiór klientów,

$J - $ zbiór potencjalnych lokalizacji węzłów pośredniczących,

$J_s\subseteq J - $ zbiór lokalizacji, które znajdują się nie dalej niż $R$ km od klienta s,

$T_j - $ zbiór typów urządzeń dostępnych w węźle pośredniczącym, gdyby go ulokowano w lokalizacji, $j$

$q_{jt} - $ liczba klientów, których urządzenie może obsłużyć;
$c_{jt} - $ koszt urządzenia.

\subsection{Funkcja celu}

\begin{equation}
min \sum_{j \in J} \sum_{t \in T_j} c_{jt}y_{jt}
\end{equation}

\subsection{Ograniczenia}

\begin{equation}
  \forall_{s \in {S}} : \sum_{j \in J_s} x_{sj} = 1
\end{equation}

\begin{equation}
  \forall_{j \in J} : \sum_{s \in S : j \in J_s } x_{sj} \leq \sum_{t \in T_j} q_{jt}y_{jt}
\end{equation}

\begin{equation}
  \forall_{j \in J} : \sum_{t \in T_j} y_{jt} \leq 1
\end{equation}

\begin{equation*}
  \forall_{s \in S} \forall_{j \in J_s} : x_{sj} \in {0,1}; \forall_{j \in J} \forall_{t \in T_j}: y_{jt} \in \mathbb_{Z_+}
\end{equation*}

\subsection{Zadanie}

\subsubsection{Znaczenie zmiennych}

\begin{itemize}
  \item $y_{jt} - $ zmienna całkowita określająca liczbę wykorzystanych urządzeń typu $t$ w lokalizacji $j$ (z (4) wynika, że jest binarna).
  \item $x_{sj} - $ zmienna binarna określająca do której lokalizacji przypisany jest klient.
\end{itemize}

\subsubsection{Interpretacja równań}

\begin{enumerate}
  \item Funkcja celu (1) to minimalizacja całkowitego kosztu potrzebnych urządzeń. Tzn sprawdzamy z których urządzeń korzystamy ($y$) i sumujemy ich koszt ($c$).
  \item Równanie (2) to ograniczenie, mówiące o tym, że każdy klient korzysta z dokładnie jednej lokalizacji.
  \item Ograniczenie (3) zostało dodane w celu upewnienia się, że liczba przpisanych klientów nie przekracza możliwości obsługi urządzenia ($x \leq q$)
  \item Ostatnie ograniczenie (4) mówi o tym, że w danej lokalizacji montowane jest co najwyżej jedno urządzenie nadawcze.
\end{enumerate}

\subsubsection{Metoda}

Jedną z metod która z których będzie korzystało dokładne rozwiązanie tego problemu jest branch-and-bound.
Polega ona na podziale problemu na podproblemy (divide-and-conquer) w taki sposób, że dla nowego utworzonego podproblemu
dodajemy nowe ograniczenie, dzieląc przestrzeń rozwiązań na dwie części (branch) . Np. zakładamy że w podproblemie "a" $x > 5$, a w podproblemie "b" $x \leq 5$.
Następnie próbujemy rozwiązać tak postawione zadanie. Jeżeli okazuje się, że tak powstała gałąź rozwiązań oferuje zawsze lepszą wartość funkcji
celu niż pozostałe rozwiązania, możemy wtedy zawęzić przestrzeń rozwiązań do nowopostałego problemu ograniczonego (bound).

Problem z tego zadania można by rozbić na dwa pod problemy w ten sposób:
zakładamy, że dla Lokalizacji pierwszej: $\sum_{t \in T_j} y_{jt} = 0$ czyli nie będzie w niej żadnego urządzenia.
I jeżeli okaże się, że cena urządzeń w tej lokalizacji jest bardzo wysoka a także możliwe jest nieskorzystanie z tej stacji,
wtedy można stwierdzić, że rozwiązanie należące do tej gałęzi będzie zawsze dawało lepszą wartość funckji celu. W konsekwencji możemy
drugą (pozostałą gałąż, gdzie: $\sum_{t \in T_j} y_{jt} = 1$) odrzucić i już nie rozpatrywać. Co oczywiście upraszcza problem.

\subsubsection{Wpływ relaksacji}
W przypadku relaksacji równania (4) możemy instalować więcej niż jedno urządzenie w danej lokalizacji. Biorąc od uwagę, że funkcja celu
opiera się wyłacznie na cenie, potencjalnie będziemy mogli użyć najtańszych urządzeń w całości systemu, czyli wartość funckji celu dla zrelaksowanego
problemu będzie nie gorsza (może być lepsza tj. mniejsza) niż w przypadku wyjściowym.

\subsubsection{Dualizacja}

\begin{equation}
  L(x, y, \lambda) = \sum_{j \in J} \sum_{t \in t_j} c_{jt}y_{jt} - \sum_{j \in J} \lambda_j (\sum_{s \in S, j \in J_s } x_{sj} - \sum_{t \in T_j} q_{jt}y_{jt})
\end{equation}

\begin{equation}
  \lambda_j \geq 0,
\end{equation}


\begin{equation}
  W(\lambda) = \min_{x, y} L(x, y, \lambda) = \min_{x, y} \sum_{j \in J} \sum_{t \in T_j} (c_{jt}y_{jt} + \lambda_jq_{jt}y_{jt})
\end{equation}


\begin{equation}
  - \sum_{j \in J} \sum_{s \in S, j \in J_s} \lambda_jx_{sj} = \min_x -\sum_{j \in J}\sum_{s \in S, j \in J_s} \lambda_sx_{sj}
\end{equation}


A biorąc pod uwagę ograniczenie (2):


\begin{equation}
  = - \sum_{s \in S} \max_{j \in J_s} \lambda_j
\end{equation}

Wobec tego dualna funkcja celu ma postać:

\begin{equation}
   \max W(\lambda) = \max - \sum_{s \in S} \max_{j \in J_s} \lambda_j
\end{equation}

Czyli po uwaględnieniu ograniczenia (6)

\begin{equation}
  \max W(\lambda) = 0
\end{equation}

\subsubsection{Liście w topologii}
 Tak, jeżeli żaden z klientów nie będzie przypisany do danego węzła pośredniczącego, czyli:
 \begin{equation}
   \exists_{j \in J} \sum_{s \in S} x_{sj} = 0
 \end{equation}
 Stopień takiego węzła, zgodnie z defnicją to 1 : $\deg j = 1$

\subsubsection{Korzeń topologii}

Centrala będzie korzeniem takiej topologi. Przy czym, zależy to od interpretacji, potencja za korzeń można by uznać dowolny wierzchołek grafu.

\subsubsection{Dane sprzeczne}

Aby sfromuowany problem był sprzeczny (nie miał rozwiązań) najłatwiej jest złamać ograniczenie (3),
to znaczy by liczba klientów przekraczała sumaryczne możliwości obsługi wszystkich urządzeń np.:

\begin{equation}
  |S| = 2
\end{equation}

oraz

\begin{equation}
  \sum_{j \in J} \sum_{t in T} q_{jt} = 1
\end{equation}

Z (6) wynika, że liczba klientów to 2, a z (7), że jesteśmy w stanie obsłużyć tylko jednego klienta. Q.E.D.

\end{document}
