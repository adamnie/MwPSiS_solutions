\documentclass{article}
\usepackage[utf8]{inputenc}

\title{Homework 5}
\author{Adam Niedziałkowski}
\date{26 December 2016}

\begin{document}

\maketitle

\section{Problem}
Projektowanie szerokopasmowej sieci dostępowej można przedstawić następująco
(zapis jest celowo nadmiarowy): w pewnej lokalizacji między cen-
tralą a grupą klientów instaluje się węzeł pośredniczący, do którego od cen-
trali doprowadza się kabel światłowodowy, a potem od niego rozprowadza
sygnał za pomocą kabli miedzianych do klientów (np. z użyciem techniki
xDSL).  Węzeł  pośredniczący  dokonuje  konwersji  optyczno-elektrycznej  i
pracuje  jako  koncentrator.  Użycie  jak  najkrótszego  segmentu  złożonego
z  kabli  miedzianych  byłoby  korzystne  dla  klienta,  ponieważ  im  krótszy
taki segment, tym większa przepływność, ale z punktu widzenia operatora
sensowne jest użycie jak nadłuższych odcinków już dawno położonej in-
frastruktury  miedzianej  (w  związku  z  użyciem  istniejącej  infrastruktury
pomijamy  tutaj  koszty  położenia  kabli).  Przy  założonej  przepływności,
którą ma uzyskać każdy klient, długość okablowania miedzianego łączą-
cego  węzeł  pośredniczący  z  klientem  nie  może  być  dłuższa  niż R km.
Z punktu widzenia topologii fizycznej sieć złożona z wierzchołków reprezen-
tujących centralę, węzeł pośredniczący (węzły pośredniczące) oraz klien-
tów jest drzewem. Problem polega na znalezieniu takiego umiejscowienia
węzłów pośredniczących obsługujących wszystkich klientów, że pojedyn-
czy  węzeł  pośredniczący  może  obsłużyć  wszystkich  przyłączonych  klientów,
tj.
\subsection{Oznaczenia}
Oznaczenia:

$S - $ zbiór klientów,

$J - $ zbiór potencjalnych lokalizacji węzłów pośredniczących,

$J_s\subseteq J - $ zbiór lokalizacji, które znajdują się nie dalej niż $R$ km od klienta s,

$T_j - $ zbiór typów urządzeń dostępnych w węźle pośredniczącym, gdyby go ulokowano w lokalizacji, $j$

$q_{jt} - $ liczba klientów, których urządzenie może obsłużyć;
$c_{jt} - $ koszt urządzenia.

\subsection{Funkcja celu}

\begin{equation}
min \sum_{j \in J} \sum_{t \in T_j} c_{jt}y_{jt}
\end{equation}

\subsection{Ograniczenia}

\Zadanie


\end{document}
