\documentclass{article}
\usepackage[utf8]{inputenc}

\title{Zadania podobne do egzaminacyjnych}
\author{Adam Niedziałkowski}
\date{16 stycznia 2016}

\begin{document}

\maketitle

\section{Zadanie 6}
$x_{ij} - $ liczba klientów przypiedzielonych do koncetratora w mieście $j$
którzy wykupili klasę obsługi $i$
P - liczba miast
N - liczba klas obsługi

\begin{equation}
  \sum_{i=2}^{N} x_{ij} = x_{1j} , j=1,2,...,P
\end{equation}

To równanie oznacza, że połowa klientów będzie osbsługiwana przez pierwszą klasę obsługi.

\section{Zadanie 7}


$N -$ zbiór komórek

$M -$ zbiór zestawów częstotliwości

$a_{ij} - $ macierz sąsiedztwa, 1 jeżeli komórka $i$ sąsiaduje z $j$, 0 w innym przypadku

$S_n - $ zbiór sąsiadów komórki $n$, inaczej dla każdego $i \in N$ takie $j \in N, j \neq i$ że $a_{ij} = 1$

$x_{nm} - $ zmienna binarna mówiąca o tym czy w komórce n wykorzystywana jest czestotliwość m

\subsection{Funkcja celu}
\begin{equation}
  \max \sum_{n=1}^N \sum_{m=1}^M (x_{nm} == 0)
\end{equation}

\subsection{Ograniczenia}

\begin{equation}
  \forall_{n \in N} \sum_{m=1}^{M} x_{nm} = 1
\end{equation}

\begin{equation}
  \forall_{n \in N} \forall_{s \in S_n} \forall_{m \in M} \quad x_{nm} + x_{sm} \le 1
\end{equation}

\subsection{Liczba zmiennych, ograniczeń}

Liczba zmiennych będzie wynosić NxM (x_{nm})

Liczba ograniczeń (3) będzie wynosić NxM.

Liczba ograniczeń (4) będzie wynosić NxNxM (technicznie to bardziej $\sum_{i,j} a_{ij} x M$, ale pierwsza odpowiedz chyba jest akceptowalna


\section{Zadanie 8}





\end{document}
