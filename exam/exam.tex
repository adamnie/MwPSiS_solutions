\documentclass{article}
\usepackage[utf8]{inputenc}

\title{Zadania podobne do egzaminacyjnych}
\author{Adam Niedziałkowski}
\date{16 stycznia 2016}

\begin{document}

\maketitle

\section{Zadanie 6}
$x_{ij} - $ liczba klientów przypiedzielonych do koncetratora w mieście $j$
którzy wykupili klasę obsługi $i$
P - liczba miast
N - liczba klas obsługi

\begin{equation}
  \sum_{i=2}^{N} x_{ij} = x_{1j} , j=1,2,...,P
\end{equation}

To równanie oznacza, że połowa klientów będzie osbsługiwana przez pierwszą klasę obsługi.

\section{Zadanie 7}


$N -$ zbiór komórek

$M -$ zbiór zestawów częstotliwości

$a_{ij} - $ macierz sąsiedztwa, 1 jeżeli komórka $i$ sąsiaduje z $j$, 0 w innym przypadku

$S_n - $ zbiór sąsiadów komórki $n$, inaczej dla każdego $i \in N$ takie $j \in N, j \neq i$ że $a_{ij} = 1$

$x_{nm} - $ zmienna binarna mówiąca o tym czy w komórce n wykorzystywana jest czestotliwość m

\subsection{Funkcja celu}
\begin{equation}
  \max \sum_{n=1}^N \sum_{m=1}^M (x_{nm} == 0)
\end{equation}

\subsection{Ograniczenia}

\begin{equation}
  \forall_{n \in N} \sum_{m=1}^{M} x_{nm} = 1
\end{equation}

\begin{equation}
  \forall_{n \in N} \forall_{s \in S_n} \forall_{m \in M} \quad x_{nm} + x_{sm} \le 1
\end{equation}

\subsection{Liczba zmiennych, ograniczeń}

Liczba zmiennych będzie wynosić NxM (x_{nm})

Liczba ograniczeń (3) będzie wynosić NxM.

Liczba ograniczeń (4) będzie wynosić NxNxM (technicznie to bardziej $\sum_{i,j} a_{ij} x M$, ale pierwsza odpowiedz chyba jest akceptowalna


\section{Zadanie 8}


\section{Zadanie 9}

$D - zbiór zapotrzebowań$

$K_l - koszt łącza$

$O_l - obciążenie łącza w Mbps$

$L - zbiór łączy$

\subsection{Funkcja celu}
\begin{equation}
\min \sum_{l \in L} O_l^2
\end{equation}


\subsection{Ograniczenia}

\begin{equation}
  \forall_{l \in L} \quad O_l \leq 4
\end{equation}

\section{Zadanie 18}

\subsection{Oznaczenia}
$M - $ zbiór lokalizacji anten
$P - $ zbiór typów anten
$BW_p - $ przepływność "w dół" anteny p
$KO_p - $ koszt instalacja anteny p
$MA_p - $ liczba dostępnych anten w magazynie
$B - $ minimalna sumaryczna przepływność

\subsection{Zmienne}
$x_{mp} - $ zmienna binarna, 1 jeżeli w lokalizacja m zainstalowana jest antena p, 0 w innych przypadku

\subsection{Funkcja celu}

\begin{equation}
  \min \sum_{m \in M} \sum_{p \in p} KO_p * x_{mp}
\end{equation}

\subsection{Ograniczenia}
Nie więcej niż 3 typy anten:

\begin{equation}
   (\sum_{p \in P}(\sum_{m \in M} x_{mp}) == 0) == 0) \leq P-3
\end{equation}

Zapewnienie minimalnej przepływności:

\begin{equation}
  \sum_{m \in M} \sum_{p \in P} BW_p * x_{mp} \geq B
\end{equation}

Co najwyżej jedna antena per lokalizacja:
\begin{equation}
  \forall_{m \in M} \sum_{p \in P} x_{mp} \leq 1
\end{equation}

Anten jest ograniczona liczba:

\begin{equation}
  \forall_{p \in P} \sum_{m \in M} x_{mp} \leq {MA}_p
\end{equation}


\section{Zadanie 19}

\subsection{Oznaczenia}
$N -$ zbiór lokalizacji

$M -$ zbiór typów anten

$BW_m -$ przepływność "w dół", realizowana przez antene m

$KO_m -$ koszt inslatacji anteny m

$MA_m -$ liczba dostępnych anten m

$BU$ całkowity dostępny budżet

\subsection{Zmienne}
$x_{nm}$ - zmienna binarna, 1 jeżeli antena m zamontowana jest w lokalizacji n inaczej 0


\subsection{Funkcja celu}

\begin{equation}

  \max \sum_{n \in N} \sum_{m \in M} BW_m * x_nm

\end{equation}

\subsection{Ograniczenia}

Jedna antena per lokalizacja:
\begin{equation}
  \forall_{n \in N} \sum_{m \in M} x_{nm} = 1
\end{equation}

Inwestor nie sra pieniędzmi:
\begin{equation}
  (\sum_{n \in N} \sum_{m \in M} x_{nm} * {KO}_m) \leq BU
\end{equation}

Anten jest ograniczona liczba:

\begin{equation}
  \forall_{m \in M} \sum_{n \in N} x_{nm} \leq {MA}_m
\end{equation}


\end{document}
